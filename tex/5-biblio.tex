\nocite{*}
%\bibliographystyle{gost780u}
%\bibliography{0-main}
\begin{thebibliography}{9} 
	
	\bibitem{selenium} Vila E., Novakova G., Todorova D. Automation Testing Framework for Web Applications with Selenium WebDriver: Opportunities and Threats // Proceedings of the International Conference on Advances in Image Processing (ICAIP 2017). 2017. P. 144-150.
	
	\bibitem{rpc_testing} Sinaga A. M., Adi Wibowo P., Silalahi A., Yolanda N. Performance of Automation Testing Tools for Android Applications // 10th International Conference on Information Technology and Electrical Engineering (ICITEE). 2018. P. 534-539.
	
	\bibitem{appium_opencv} Mozgovoy M., Pyshkin E. Unity Application Testing Automation with Appium and Image Recognition // In: Itsykson V., Scedrov A., Zakharov V. (eds) Tools and Methods of Program Analysis. TMPA 2017. Communications in Computer and Information Science. 2017. №779. P. 139--150.
	
	\bibitem{wimp} Andries van Dam Post-WIMP user interfaces // Commun. ACM. 1997. №2. P. 63–67.
	
	\bibitem{habr} Лучший игровой движок по версии пользователей хабра // Хабр URL: https://habr.com/ru/post/307952/ (дата обращения: Апрель 11, 2020).
	
	\bibitem{disser} Кугуракова В.В МАТЕМАТИЧЕСКОЕ И ПРОГРАММНОЕ ОБЕСПЕЧЕНИЕ МНОГОПОЛЬЗОВАТЕЛЬСКИХ ТРЕНАЖЕРОВ С ПОГРУЖЕНИЕМ В ИММЕРСИВНЫЕ ВИРТУАЛЬНЫЕ СРЕДЫ: дис. ... канд. тех. наук: 05.13.11. Казань, 2019. 187 с.
	
	\bibitem{vr-simulators} Kugurakova V.V. Automated approach to creating multi-user simulators in virtual reality // CEUR Workshop Proceedings. 2018. №2260. P. 313-320.  
	
	\bibitem{visual-editor} Kugurakova V.V., Abramov V.D., Abramsky M.M., Monks N., Maslaviev A. Visual editor of scenarios for virtual laboratories // DeSE 2017, Developments in the design of electronic systems.. Paris: Conference Publication Services (CPS), 2017. P. 242-247.
	
	%%% Вступление остортировано
	
	\bibitem{unity_input_systems} Unity -- Manual: Input // Unity Manual URL: https://docs.unity3d.com/Manual/Input.html (дата обращения: Апрель 12, 2020).
	
	\bibitem{solid} Мартин Р.С., Ньюкирк Д.В., Косс Р.С. Быстрая (гибкая) разработка программ на Java и C++: принципы, шаблоны, практика. М.: Издательский дом ``Вильямс'', 2001. 752 с.

	\bibitem{dry} Haoyu W., Haili Z. Basic Design Principles in Software Engineering // Fourth International Conference on Computational and Information Sciences. Chongqing: IEEE Computer Society, 2012. P. 1251-1254.
	
	\bibitem{razor} Blumer A., Ehrenfeucht A., Hasler D., Warmuth M.K. Occam's Razor // Information Processing Letters 24. 1987. №6. P. 377-380.
	
	\bibitem{dots} High-performance multithreaded stack of unity information-oriented technologies // unity.com URL: https://unity.com/ru/dots (дата обращения: Июль 19, 2019).
	
	\bibitem{ecs} Unity Foundation -- Introduction to ECS // unity3d.com URL: https://unity3d.com/ru/learn/tutorials/topics/scripting/introduction-ecs (дата обращения: Июль 19, 2019).
	
	\bibitem{interceptor} Schmidt D., Stal M., Rohnert H., Buschmann F. Pattern-Oriented Software Architecture // Patterns for Concurrent and Networked Objects. 2001. №2. P. 109-140.
	
	\bibitem{broker} Architectural pattern ``Broker'' // cs.uno.edu URL: http://cs.uno.edu/~jaime/Courses/4350/broker.ppt (дата обращения: Июль 19, 2019).
	
	\bibitem{pac} Coutaz J. PAC: AN OBJECT ORIENTED MODEL FOR IMPLEMENTING USER INTERFACES // ACM SIGCHI Bulletin. 1987. №19. P. 37-41.
	
	\bibitem{jsr} JSR 299: Contexts and Dependency Injection for the JavaTM EE platform // jcp.org URL: https://jcp.org/en/jsr/detail?id=299 (дата обращения: Июль 19, 2019).
	
	\bibitem{oreilly} Брюс П., Брюс Э. Практическая статистика для специалистов Data Science. СПб.: БХВ-Петербург, 2018. 304 с.
	
	
\end{thebibliography}