\Introduction
\section*{Постановка проблемы}
В жизненном цикле разработки каждого программного обеспечения (ПО) присутствует этап приемо-сдаточных работ. Перед тем как передавать ответственному за этот этап уставному органу проекта собранное ПО, необходимо провести один из завершающих этапов разработки, а именно регрессионное тестирование. Алгоритм проведения данного вида тестирования технически довольно прост: создаются сценарии пользовательского поведения в виде диаграммы вариантов использования, выбирается формальный исполнитель, который шаг за шагом выполняет действия, описанные в главных, а затем альтернативных потоках вариантов использования.

Вопрос выбора формального исполнителя здесь стоит довольно остро, так как это напрямую влияет на стоимость разработки. Для разных групп используемых технологий разработки данная проблема решается по-разному. Например, для популярных платформ, таких как PC, Web, iOS, Android, уже существуют разные автоматические формальные исполнители для приложений с различными возможностями и характеристиками, для однократной записи и множественного проигрывания действий потоков. Ими, например, являются Selenium, Appium, Robotium и UI Automator. Формы решения вопроса выбора исполнителя существуют разные, например, для Web используется WebDriver для записи и имитации ввода \cite{selenium}, для iOS и Android используется система RPC вызовов \cite{rpc_testing}, если же нет возможности вмешаться в поток ввода для облегчения записи, то используется компьютерное зрение для распознавания того, как и с чем взаимодействует пользователь \cite{appium_opencv}. Метод зачастую довольно эффективный для графических систем с конечным набором однотипных элементов (напр. класс систем Windows, Icons, Menus, Pointers (WIMP) \cite{wimp}), однако гораздо менее точный при большом разнообразии динамических графических элементов. В дополнение вышеописанные исполнители для данных платформ ограничены устройствами ввода, а именно клавиатурой и мышкой.

Особняком здесь стоит задача о выборе формального исполнителя для тестирования игровых приложений на популярной платформе для разработки приложений визуализации и игр \textit{Unity} \cite{habr}. Каждая из вышеупомянутых форм решения вопроса в этом случае проигрывает и в эффективности, и в степени удобства интеграции инструмента записи и проигрывания действий, так как зачастую игры насыщены динамическими графическими элементами, а некоторые игры вообще не имеют графический интерфейс. В отличие от других платформ, \textit{Unity} -- не ограничена устройствами ввода. Поэтому часто формальным исполнителем здесь становится группа специалистов или других людей, которые играют роль тестировщиков. Им, частично или полностью, выдаётся информация о действиях внутри потоков и они их исполняют, составляя в заключении необходимые отчёты. В данном случае стоимость разработки также повышается ввиду создания и поддержки системы контроля работы тестировщиков.

Альтернативный подход к регрессионному тестированию на платформе \textit{Unity} является автоматизированное модульное тестирование. Однако, при использовании данного подхода необходимо для каждого основного потока писать комплекс из юнит-тестов (алгоритмов тестирования описанных в коде ПО и исполняемых в контексте с ним средствами выбранной платформы). Время написания подобного комплекса примерно равно объёму времени этапа основной разработки при этом появляется необходимость поддержки в дополнении к основной кодовой базе ещё и юнит-тестов со всеми вытекающими издержками.

\section*{Цель и задачи разработки}
Целью данной работы является реализация и применение формы решения описанной выше проблемы регрессионного тестирования для игровых приложений платформы \textit{Unity} в виде импортируемой группы ассетов (исходных файлов кода и пред заготовленных графических и модульных инструментов).

Реализуемое решение основано на вышеописанных подходах для платформ PC, Web и др., и на возможности декорирования методов взятия ввода, обусловленной внутренней архитектурой платформы. Две главных особенностей данного решения -- это высокая точность распознавания и перехвата ввода и возможность избежать регулярных потерь времени при человеческом тестировании за счёт однократной записи ввода не только с клавиатуры и компьютерной мыши, но и любого другого периферийного устройства и дальнейшего проигрыша записанного ввода по необходимости регрессионного тестирования.

В настоящей выпускной квалификационной работе рассматриваются задачи, связанные напрямую с исследованием применения инструментария взятия ввода, а также задачи цикла разработки данных ассетов на платформе \textit{Unity}.

В первой главе рассматриваются исследование среднего количества использований внутреннего инструмента взятия ввода платформы \textit{Unity} среди самых значимых и популярных проектов на данной платформе. Происходит анализ класса совместимых приложений с описанными в данной работе ассетами, а также определение теоретической базы приёмов программной инженерии, которые были использованы в рамках данной работы.

Во второй главе описан процесс разработки описываемой группы ассетов, состоящий из трёх подпроцессов: реализация вспомогательных систем для статистических исследований и управления зависимостями, создание базовой системы интеграции ассетов и реализация основных систем, позволяющих осуществить запись и проигрывание, хранение, загрузку, экспорт и импорт действий.

Дополнительная задача при разработке данной группы ассетов, описанная в конце второй главы -- возможность сериализовывать, экспортировать и импортировать записанные данные для последующего использования. Например, для построения сценария для тренажеров в виртуальной реальности \cite{disser} или для генерации сценария обучающего тренажера \cite{vr-simulators}, что поможет избежать рутинной работы по формированию треков обучения \cite{visual-editor}.

Третья глава посвящена циклу разработки окон управления группой ассетов после интеграции в сторонний проект, а также разработке и развёртыванию онлайн-документации, в которой описывается процесс использования окон управления. Здесь же описываются принципы дополнения и расширения исходной кодовой базы описываемых ассетов для всех желающих сторонних разработчиков.

Четвертая и заключительная глава рассматривает опыт портирования и применения разработанной группы ассетов внутри проекта виртуальной биотехнологической лаборатории \cite{biolab}.

\section*{Объект и предмет разработки}
Объектом разработки является процесс автоматизации регрессионного тестирования для приложений с богатой или отсутствующей динамической графикой и пользовательским интерфейсом на платформе \textit{Unity}. В качестве предмета разработки в данной работе описывается группа ассетов, предназначенных для автоматизации действий пользователя.

Иными словами, результатом представлен набор дополнительных окон и утилит для редактора \textit{Unity} для записи и проигрывания сценариев, автоматической интеграции группы ассетов в готовую кодовую базу стороннего проекта, а также для сохранения и загрузки записанных сценариев, представляющих собой входные данные со всех периферийных устройств, что были активны во время работы приложения, в которое группа ассетов была импортирована.

Дополнением к результату является документация к пользованию разработанными ассетами, их расширению для нужд сторонних разработчиков, а также видеофайл, демонстрирующий работу с симулятором VR устройства в проекте DML BioLab \cite{biolab}, где происходит запись и проигрывание первого пункта первого сценария.