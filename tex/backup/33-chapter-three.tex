\chapter{Человеко-машинный интерфейс}

В рамках данной работы человеко-машинный интерфейс представлен в виде пользовательских окон управления в Unity Editor, а также в виде онлайн-ресурса посвящённого обучению и модификации описываемой работы.

\label{cha:ch_3}
При создании окон управления не были использованы ни синглтон заготовки, ни написанный DI контейнер ввиду особенностей распределения времени выполнения у \textit{Unity} -- пространство имен, взаимодействующее с классами, которые регламентируют внешний вид и поведение пользовательских окон, изолировано от других пространств, что делает невозможным использование DI и других вспомогательных структур. 

\section{Окно управления записью и проигрыванием}
Данное окно (см. рис. \ref{recorderUI}) состоит из трёх секций: указание текущей реализации проигрывателя, состояние проигрывателя и элементы управления проигрыванием и записью.

\begin{figure}[h]
	\centering
	\includegraphics[width=0.7\linewidth]{recorder.PNG}
	\caption{Editor UI для проигрывателя}
	\label{recorderUI}
\end{figure}

В процесс управления записью входит:
\begin{itemize}
	\item
	просмотр текущего состояния проигрывателя по чекбоксам состояния на рис.\ref{recorderUI};
	\item
	определение имени записи, которую нужно проиграть или под которое нужно записать поток действий;
	\item
	непосредственно сами кнопки управления записью -- Start, Stop, Play, Continue;
	\item
	аналогичные кнопки управления проигрыванием -- Start, Stop, Play, Continue.
\end{itemize}
Для того чтобы проиграть выбранную запись, необходимо сначала вписать нужное имя в поле \textit{Name of the recording} или же просто щёлкнуть правой кнопкой мыши по записи в активной зоне окна управления хранилищем.

Также, была создана функция остановки ввода. Для этого необходимо нажать комбинацию клавиш ``Ctrl+I''. Мотивация к созданию функционала по остановке ввода, была в необходимости, не используя мышку, в полной мере пользоваться ATF. Существует множество приложений визуализации и игр, разработанных на платформе \textit{Unity}, которые используют мышку во время исполнения. Без возможности останавливать ввод, было бы невозможно использовать любые окна управления ATF, оставляя при этом неизменным состояние приложения.

В дополнение к возможностям управления без мышки были добавлены новые комбинации клавиш (см. рис. \ref{shortcuts}).

\begin{figure}[h]
	\centering
	\includegraphics[width=\linewidth]{shortcuts.png}
	\caption{Комбинации клавиш управления окном записи}
	\label{shortcuts}
\end{figure}

Данные комбинации работают также как переключатели света. Другими словами, первое нажатие активирует функцию, второе -- выключает её.
\begin{itemize}
	\item ``Ctrl+U'' -- Включение или выключение записи.
	\item ``Ctrl+Shift+U'' -- Пауза или продолжение записи.
	\item ``Ctrl+Y'' -- Включение или выключение проигрывания.
	\item ``Ctrl+Shift+Y'' -- Пауза или продолжение проигрывания.
\end{itemize}

\section{Окно управления хранилища действий}
Окно управления хранилищем (см. рис. \ref{storageUI}) также состоит из трёх секций -- информация о реализации, секции управления сохранением и загрузкой, а также зоны иллюстрации текущего состояния хранилища.

\begin{figure}[h]
	\centering
	\includegraphics[width=0.7\linewidth]{storage.PNG}
	\caption{Editor UI для хранилища действий}
	\label{storageUI}
\end{figure}

В процесс управления хранилищем входит:
\begin{itemize}
	\item
	просмотр текущей записи хранилища;
	\item
	определение имени записи, которую нужно загрузить или сохранить; 
	\item
	непосредственно сами кнопки управления сохранением -- Save, Load, Scrap (удалить);
	\item
	загрузка в активную зону записей из реестра (Окна: Current records, Current commands and actions queues);
	\item экспорт и импорт записей в пассивной зоне на внешний накопитель.
\end{itemize}
Зоны иллюстрации хранилища делятся на активную и пассивную. Активная зона хранит в себе те записи, которые находятся сейчас непосредственно в оперативной памяти, пассивная же зона хранит в себе записи, которые были сохранены на внешнем накопителе или реестре.

Чтобы экспортировать всё, что находится в пассивной зоне, необходимо вписать абсолютный путь к файлу с расширением ``.json'', где и будет располагаться сохранённое хранилище. Далее необходимо нажать на кнопку \textit{Export saved storage}.

Для выполнения импорта хранилища, нужно проделать те же шаги, но путь указать к уже существующему файлу. Далее следует нажать на кнопку \textit{Import saved storage}. 

\section{Окно управления интеграцией}

Окно управления интеграцией состоит из трёх секций и работает в двух режимах интеграции и в двух режимах выбора путей. Первая секция -- информация об используемой реализации, вторая -- секция добавления и удаления путей к файлам исходного кода и третья -- секция с отображением выбранных файлов и кнопками управления.

Находясь в первом режиме выбора путей (см. рис. \ref{integratorUI}), пользователь вводит пути к файлам исходного кода, в которые следует интегрировать ATF, вручную. Вводимый путь является относительным и его корень лежит в папке Assets, в директории проекта. 

\begin{figure}[h]
	\centering
	\includegraphics[width=0.7\linewidth]{integrator_manual.PNG}
	\caption{Editor UI для управления интеграцией в ручном режиме}
	\label{integratorUI}
\end{figure}

Второй режим выбора путей (см. рис. \ref{integratorUIAutomatic}) позволяет упростить и ускорить сбор путей к файлам исходного кода. Для этого нужно отключить чекбокс \textit{Manual path choosing}. После его отключения, файлы исходного кода можно выделять мышкой во вкладке Project в Unity Editor. Все выбранные файлы автоматически попадают в очередь на интеграцию.

\begin{figure}[H]
	\centering
	\includegraphics[width=0.7\linewidth]{integrator_automatic.PNG}
	\caption{Editor UI для управления интеграцией в автоматическом режиме}
	\label{integratorUIAutomatic}
\end{figure}

Удаление путей в обоих режима происходит одинаково: выделение пути в списке и нажатие на кнопку \textit{Remove path}. Стоит заметить, что, если не убрать полностью выделение во вкладке Project, курсор в виде текущего выбранного пути, отображаемого в секции \textit{Current path}, будет обращён на последний выделенный путь, из-за чего при невнимательном управлении можно случайно удалить не те пути, что были выбраны в списке.

Так как система интеграции поддерживает два режима работы: автоматический и полуавтоматический, при проектировании данного окна было решено выделить отдельные кнопки для обоих режимов:
\begin{itemize}
	\item
	Кнопки \textit{Integrate} и \textit{Integrate and replace} отвечают за полуавтоматический режим и воздействуют на представленные в третьей секции конкретные файлы исходного кода.
	\item
	Кнопка \textit{Integrate All} отвечает за использование автоматического режима.
\end{itemize}

Кнопки \textit{Save paths} и \textit{Load paths} отвечают за сохранение и загрузку представленных в третьей секции файлов исходного кода. Для облегчения работы с файлами, вместо конкретных файлов во второй секции было решено использовать пути файлов.

\section{Онлайн-документация и способ доступа}
Для лучшего понимания функционала разработанной группы ассетов была создана онлайн-документация \cite{atf_docs}. Данный ресурс содержит в себе учебный материал не только для обучения пользованием ассетами, но и информацию о каждом интерфейсе основной системы и учебный материал, где рассказывается о том, как можно модифицировать разработанные ассеты, а также создавать новые. 

В качестве платформы для создания онлайн-документации было решено использовать ReadTheDocs.org (RTD) с генератором документации Sphinx (см. рис. \ref{online_docs}). Критерием к выбору платформы и генератора служила возможность бесплатного хостинга ресурса с адекватным на мнение автора именем хоста, лёгкость языка шаблонизатора у генератора, а также высокий уровень репутации ресурса.

Процесс развёртывания данной онлайн-документации на платформе RTD заключается всего в двух шагах: инициализация в сервисе хостинга github-репозитория с исходным кодом статей на языке ReST и запрос на автоматическую сборку и развёртывание онлайн-документации.

\begin{figure}[H]
	\centering
	\includegraphics[width=\linewidth]{online_docs.png}
	\caption{Главная страничка онлайн-документации}
	\label{online_docs}
\end{figure}

Это было необходимо, чтобы не отпугнуть потенциального контрибьютора от изучения документации. Возможности контрибьютинга (совместной разработки с публикой) обусловлены тем, что лицензия результатов данной разработки создана на основе MIT.

В дополнение к созданной онлайн-документации, была добавлена возможность быстрого поиска по статьям, представленным в ней. В качестве языка шаблонов был использован reStructuredText -- облегчённый язык разметки схожий по синтаксису с языком Markdown, однако позволяющий автоматически генерировать файлы PDF, HTML, ODT, LaTeX, а также формат презентаций S5.

Так как разработанная группа ассетов распространяется по лицензии MIT, доступ к исходному коду открыт и доступен на сайте GitHub.com \cite{git}.

В руководство для пользователей (разработчиков проекта, в который ATF был интегрирован) входит множество инструкций по использованию ATF (см. рис. \ref{user_online_docs}). 

Первая инструкция посвящена интеграции ATF в новый пустой проект. Там иллюстрируюется способ доступа к свежему пакету с ATF, а также объясняется как подготовить сцену и где найти управляющие окна. Вторая инструкция демонстрирует интеграцию в существующий проект и аспекты использования статического или объектного доступа к перехатчику. Третья и четвертая инструкции иллюстрирует работу всех режимов перехватчика и окна записи соответственно. Здесь объясняется то, как можно использовать механизм остановки ввода, не нарушая состояния приложения во время исполнения. Пятая инструкция посвящена функционалу манипулирования хранилищем действий, описывая его зоны для просмотра содержания записи и функции его импорта и экспорта. Заключительная шестая инструкция объясняет как пользоваться комбинациями клавиш для управления записью.

\begin{figure}[H]
	\centering
	\includegraphics[width=0.3\linewidth]{user_guide.PNG}
	\caption{Содержание руководства для пользователей}
	\label{user_online_docs}
\end{figure}

В руководство для контрибьюторов (см. рис. \ref{dev_online_docs}) входят обзорные статьи посвященные принципам построения кодовой базы ATF.

Первая статья посвящена созданию новой системы, которая бы автоматически инициализировалась на сцене, сразу же после инъекции зависимостей. Вторая статья посвящена обзору платформы ATF и тому, как организовано взаимодействие со стандартными системами \textit{Unity}. Заключительная статья содержит описания всех интерфейсов основных систем ATF, включая комментарии по хранению данных и модификации.

\begin{figure}[H]
	\centering
	\includegraphics[width=0.3\linewidth]{dev_guide.PNG}
	\caption{Содержание руководства для контрибьюторов}
	\label{dev_online_docs}
\end{figure}
