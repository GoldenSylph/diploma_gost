\chapter{Листинги}
\section{Исходный код класса обработки веб-страниц проектов с сайта GitHub.com}
\begin{lstlisting}[caption={Исходный код класса обработки веб-страниц проектов с сайта GitHub.com},label=spider]

# -*- coding: utf-8 -*-
import scrapy
import requests
import json
import base64
import re
import time
import pandas as pd
class GithubSpiderSpider(scrapy.Spider):

	name = 'github-spider'
	main_data = pd.DataFrame(columns=['repo_name', 'method_invocations'])
	file_name = 'main_data_most_forks.csv'
	config = False
	recently_updated_url = 
	'https://github.com/search?o=desc&p={}
		&q=unity+language%3AC%23&s=updated&type=Repositories'
	most_stars_url = 
	'https://github.com/search?o=desc&p={}
		&q=unity+language%3AC%23&s=stars&type=Repositories'
	most_forks_url = 
	'https://github.com/search?o=desc&p={}
		&q=unity+language%3AC%23&s=forks&type=Repositories'
	best_match_url = 
	'https://github.com/search?p={}
		&q=language%3Acsharp+unity&type=Repositories'

	def start_requests(self):
		urls = [self.most_forks_url.format(i + 1) for i in range(101)]
		for url in urls:
			yield scrapy.Request(url=url, callback=self.parse)
	
	def auth_data(self):
		if self.config:
			return (self.config['login'], self.config['password'])
		with open('github_config.json', 'r', encoding='utf-8') as github_file:
			self.config = json.load(github_file)
			return (self.config['login'], self.config['password'])
		
	def get(self, url):
		result = requests.get(url, auth=self.auth_data())
		if result.status_code == 403:
			header_name = 'X-RateLimit-Reset'
			if header_name in result.headers:
				reset_time = float(result.headers[header_name])
				sleep_time = reset_time - time.time()
				if sleep_time < 0:
					self.logger.info('Sleep time is negative, adjusting to 1...')
					sleep_time = 1
				self.logger.info('Exceeded rate limit. Waiting for: {}'.format(sleep_time))
				time.sleep(sleep_time)
				result = requests.get(url, auth=auth_data)
			else:
				return False
				self.logger.info('{}, status: {}'.format(result.headers, result.status_code))
		return result
	
	def get_json(self, response):
		try:
			return response.json()
		except:
			self.logger.info(traceback.format_exc())
			return False
	
	def closed(self, reason):
		self.logger.info('Done. Specified reason: {}'.format(str(reason)))
	
	def parse(self, response):
		repo_names = [r[1:] for r in response.xpath('//div[@class="f4 text-normal"]/a/@href').extract()]
		for repo_name in repo_names:
		self.logger.info('Working with repo: {}'.format(repo_name))
		code_search_url = 'https://api.github.com/search/code?
			q=Input+in:file+language:csharp+repo:{}'
			.format(repo_name)
		code_response = self.get(code_search_url)
		if not code_response:
			self.logger.info('something wrong with header, continue...')
			continue
		code_data = self.get_json(code_response)
		if not code_data:
			continue
		method_invokations_count = 0
		code_urls = [code_item['url'] for code_item in code_data['items']]
		for code_url in code_urls:
			self.logger.info('Source url: {}'.format(code_url))
			source_details_response = self.get(code_url)
			if not source_details_response:
				self.logger.info('something wrong with header in source details response, continue...')
				continue
			source_details_data = self.get_json(source_details_response)
			if not source_details_data:
				continue
			if 'message' in source_details_data:
				self.logger.info(source_details_data['message'])
				continue
			try:
				source = 
				base64.b64decode(source_details_data['content']
					.encode('utf-8')).decode('utf-8')
			except:
				self.logger.info(traceback.format_exc())
				continue
			method_invokations_count_in_source = 
				len(re.findall(r'[(\[(\s,=\+\-\*/\?&|]Input\.', source))
			method_invokations_count += method_invokations_count_in_source
		new_data = pd.DataFrame({'repo_name': repo_name, 'method_invocations': method_invokations_count}, index=[0])
		self.main_data = self.main_data.append(new_data, ignore_index=True)
		self.logger.info('Data gathered:\n{}'.format(new_data))
		self.main_data.to_csv(self.file_name)
\end{lstlisting}
\newpage
\section{Утилита для сбора собранных spider-программой данных воедино}
\begin{lstlisting}[caption={Утилита для сбора собранных spider-программой данных воедино},label=form_main_data]
import pandas as pd
import seaborn as sns
import matplotlib.pyplot as plt
sns.set(style="white", context="talk")
whole_data = pd.DataFrame(columns=['repo_name', 'method_invocations', 'filter'])
metadata = [('main_data_best_match.csv', 'best_match'),
('main_data_most_stars.csv', 'most_stars'),
('main_data_most_forks.csv', 'most_forks'),
('main_data_recently_updated.csv', 'recently_updated')]
for data in metadata:
	new_data = pd.read_csv(data[0], index_col=0)
	new_data['filter'] = data[1]
	whole_data = whole_data.append(new_data, ignore_index=True)
whole_data['filter'] = whole_data['filter'] + whole_data.duplicated(subset='repo_name').apply(str)
print(whole_data)
whole_data.to_csv('main_data.csv')
sns.scatterplot(x=whole_data.index, y='method_invocations', data=whole_data, hue='filter')
plt.show()
plt.clf()
\end{lstlisting}
\newpage
\section{Утилита для проведения бутстрапа и параллельных расчётов границ интервала среднего}
\begin{lstlisting}[caption={Утилита для проведения бутстрапа и параллельных расчётов границ интервала среднего},label=bootstrap_experiment]

import math as m
import pandas as pd
import multiprocessing as mp
import seaborn as sns
import matplotlib.pyplot as plt
import matplotlib

def stat(x):
return round(x.mean(), 3)

if __name__ == '__main__':

	whole_data = pd.read_csv('main_data.csv')
	
	R = 99509
	n = len(whole_data)
	confidence = 90
	specific_data = whole_data['method_invocations']
	
	process_count = mp.cpu_count()
	chunk_size = 1000
	print('Processes: {}, chunksize: {}'.format(process_count, chunk_size))
	
	with mp.Pool(process_count) as pool:
		boot_data =
		pd.Series(pool.map(stat, 
			[specific_data.sample(n, replace=True) for i in range(R)], 
				chunksize=chunk_size))
	
	boot_data_len = len(boot_data)
	cut_percent = m.ceil((1 - confidence / 100) / 2)
	cut = round(boot_data_len * cut_percent / 100)
	middle_cuts = [cut, boot_data_len - cut]
	tail_cut = boot_data_len - cut
	print('boot_data_len: {}, cut_percent: {}, cut: {}, middle_cuts: {}, tail_cut: {}'.format(boot_data_len, cut_percent, cut, middle_cuts, tail_cut))
	head = boot_data[:cut]
	middle = boot_data[middle_cuts[0]:middle_cuts[1]]
	tail = boot_data[tail_cut:]
	main_data = pd.DataFrame(columns=['stat', 'kind'])
	main_data = main_data.append(pd.DataFrame({'stat': head, 'kind': 'head'}), ignore_index=True)
	main_data = main_data.append(pd.DataFrame({'stat': middle, 'kind': 'interval'}), ignore_index=True)
	main_data = main_data.append(pd.DataFrame({'stat': tail, 'kind': 'tail'}), ignore_index=True)
	sns.jointplot(x=main_data.index, y=main_data['stat'], kind='hex')
	plt.axhline(y=main_data.at[cut, 'stat'], color='r')
	plt.axhline(y=main_data.at[tail_cut, 'stat'], color='r')
	plt.text(cut, main_data.at[cut, 'stat'] + 0.01, 'L: {}'.format(main_data.at[cut, 'stat']), fontsize=12)
	plt.text(cut, main_data.at[tail_cut, 'stat'] + 0.01, 'U: {}'.format(main_data.at[tail_cut, 'stat']), fontsize=12)
	plot_backend = matplotlib.get_backend()
	mng = plt.get_current_fig_manager()
	if plot_backend == 'TkAgg':
		mng.resize(*mng.window.maxsize())
	elif plot_backend == 'wxAgg':
		mng.frame.Maximize(True)
	elif plot_backend == 'Qt4Agg':
		mng.window.showMaximized()
	plt.savefig('experiment.png')
	plt.show()


\end{lstlisting}
\newpage
\section{Интерфейс модуля загрузчика хранилища записей}
\begin{lstlisting}[caption={Интерфейс модуля загрузчика хранилища записей},label=iSaver]
public interface IATFActionStorageSaver : IATFInitializable
{
	void SaveRecord();
	void LoadRecord();
	void ScrapRecord();
	
	IEnumerable GetActions();
	void SetActions(IEnumerable actionEnumerable);
	List<TreeViewItem> GetSavedNames();
	List<TreeViewItem> GetSavedRecordDetails(string recordName);
}
\end{lstlisting}
\newpage
\section{Интерфейс модуля хранилища записей}
\begin{lstlisting}[caption={Интерфейс модуля хранилища записей},label=iStorage]
public interface IATFActionStorage : IATFInitializable
{
	object GetPartOfRecord(FakeInput kind, object fakeInputParameter);
	void Enqueue(string recordName, FakeInput kind, object fakeInputParameter, AtfAction atfAction);
	AtfAction Dequeue(string recordName, FakeInput kind, object fakeInputParameter);
	AtfAction Peek(string recordName, FakeInput kind, object fakeInputParameter);
	bool PrepareToPlayRecord(string recordName);
	void ClearPlayStorage();
	void SaveStorage();
	void LoadStorage();
	void ScrapSavedStorage();
	List<TreeViewItem> GetSavedRecordNames();
	List<TreeViewItem> GetCurrentRecordNames();
	List<TreeViewItem> GetCurrentActions(string recordName);
	List<TreeViewItem> GetSavedActions(string recordName);
}
\end{lstlisting}
\newpage
\section{Бизнес-логика MonoSingleton<T>}
\begin{lstlisting}[caption={Бизнес-логика MonoSingleton<T>},label=monoSingleton]
public class MonoSingleton<T> : MonoBehaviour where T : MonoBehaviour
{
	private static T s_Instance;
	private static bool s_IsDestroyed;
	public static T Instance
	{
	get
	{
	if (s_IsDestroyed)
	return null;
	if (s_Instance == null)
	{
	s_Instance = FindObjectOfType(typeof(T)) as T;
	
	if (s_Instance == null)
	{
	var gameObject = new GameObject(typeof(T).Name); 
	DontDestroyOnLoad(gameObject);
	s_Instance = gameObject.AddComponent(typeof(T)) as T;
	}
}
return s_Instance;
...
}
\end{lstlisting}
\newpage
\section{Метод инъекции зависимостей в экземпляр класса типа}
\begin{lstlisting}[caption={Метод инъекции зависимостей в экземпляр класса типа},label=injectType]
public static void InjectType(Type t)
{
	if (!ContainsAnyAttributeOfType(
	t.GetCustomAttributes(false), typeof(InjectableAttribute))) return;
		foreach (var fi in t.GetFields())
	{
	...
	if (temp.ComponentType == null && isScenePathEmpty) temp.ComponentType = fi.FieldType;
	...
	if (!isScenePathEmpty)
	{
		var hierarchyAndComponent = GetHierarchyPathAndComponentName(temp.ScenePath);
		if (!hierarchyAndComponent.Valid)
		{
			... continue;
		} 
		gameObjectContainingObjectToInject = GameObject.Find(hierarchyAndComponent.Result[0]);
		if (!gameObjectContainingObjectToInject)
		{
			... continue;
		}
		objectToInject = gameObjectContainingObjectToInject
		.GetComponent(hierarchyAndComponent.Result[1]);
	}
	else
	{
		objectToInject = FindObjectOfType(temp.ComponentType);
		if (objectToInject)
		{
			gameObjectContainingObjectToInject = GameObject.Find(objectToInject.name);
		}
		if (!gameObjectContainingObjectToInject && !temp.LookInScene)
		{
			objectToInject = Activator.CreateInstance(temp.ComponentType) as UnityEngine.Object;
			if (!objectToInject)
			{
				... continue;
			} ...
		}
	}
	if (!gameObjectContainingObjectToInject && objectToInject == null && temp.LookInScene)
	{
	... continue;
	}
	...
	dynamic typeToWhichInjected = FindObjectOfType(t.GetTypeInfo());
	if (typeToWhichInjected == null)
	{
	... continue;
	}
	fi.SetValue(typeToWhichInjected, objectToInject);
	...
}
\end{lstlisting}
\newpage
\section{Бизнес-логика базовой системы интеграции}
\begin{lstlisting}[caption={Бизнес-логика базовой системы интеграции},label=initializer]
public class ATFInitializer : MonoBehaviour
{
	private void Awake()
	{
		IATFInitializable[] ALL_SYSTEMS = {
			ATFQueueBasedRecorder.Instance, 
			ATFDictionaryBasedActionStorage.Instance,
			ATFPlayerPrefsBasedActionStorageSaver.Instance
		};
		foreach (IATFInitializable i in ALL_SYSTEMS)
		{
			i.Initialize();
		}
		DependencyInjector.Instance.Initialize("ATF");
		DependencyInjector.Instance.Inject();
	}
}
\end{lstlisting}
\newpage
\section{Бизнес-логика системы симуляции и перехвата ввода}
\begin{lstlisting}[caption={Бизнес-логика системы симуляции и перехвата ввода},label=input]
[Injectable]
[Serializable]
[AtfSystem]
public class ATFInput : MonoSingleton<T>, IBaseInput
{
[Inject(typeof(ATFQueueBasedRecorder))]
public static readonly IATFRecorder RECORDER;
[Inject(typeof(ATFDictionaryBasedActionStorage))]
public static readonly IATFActionStorage STORAGE;
[Inject(typeof(BaseInput))]
public static readonly BaseInput BASE_INPUT;
...
private static object RealOrFakeInputOrRecord(object realInput, object fakeInput, object fip, FakeInput kind)
{ ... }
private static object GetCurrentFakeInput(FakeInput inputKind, object fip)
{
return STORAGE.GetPartOfRecord(inputKind, fip);
}
...
private static T Intercept<T>(T realInput, FakeInput fakeInputKind, object 	fakeInputParameter = null)
{
    if (fakeInputParameter == null)
    {
        fakeInputParameter = "EMPTY FAKE INPUT PARAMETER";
    }
    dynamic result = RealOrFakeInputOrRecord(realInput, fakeInputParameter, fakeInputKind);
    if (result is string)
    {
        Debug.Log(result);
    }
    return result;
}
public static bool anyKeyDown => Intercept(Input.anyKeyDown, FakeInput.ANY_KEY_DOWN, false);
...
}
\end{lstlisting}
\newpage
\section{Интерфейc модуля записи}
\begin{lstlisting}[caption={Интерфейc модуля записи},label=iRecorder]
using ATF.Scripts.Helper;
namespace ATF.Scripts.Recorder
{
    public interface IAtfRecorder : IAtfGetSetRecordName
    {
        bool IsRecording();
        bool IsPlaying();
        bool IsRecordingPaused();
        bool IsPlayPaused();
        bool IsInputStopped();
        void PlayRecord();
        void PausePlay();
        void ContinuePlay();
        void StopPlay();
        void StartRecord();
        void PauseRecord();
        void ContinueRecord();
        void StopRecord();
        void SetRecording(bool value);
        void SetPlaying(bool value);
        void SetRecordingPaused(bool value);
        void SetPlayPaused(bool value);
        void SetInputStopped(bool value);
        void Record(FakeInput kind, object input, object fakeInputParameter);
        object GetLastInput(FakeInput kind, object fakeInputParameter);
        void SetLastInput(FakeInput kind, object realInput, object fakeInputParameter); ...
\end{lstlisting}
\newpage
\section{Пользовательский модуль ввода для ATFInput}
\begin{lstlisting}[caption={Пользовательский модуль ввода для ATFInput},label=inputModule]
namespace ATF
{
	public class ATFStandaloneInputManager : StandaloneInputModule
	{
		protected override void Start()
        {
            base.Start();
            m_InputOverride = AtfInput.BASE_INPUT;
            DependencyInjector
            	.InjectType(m_InputOverride.GetType());
        }
...
\end{lstlisting}
\newpage
\section{Интерфейс модуля интеграции в готовую кодовую базу}
\begin{lstlisting}[caption={Интерфейс модуля интеграции в готовую кодовую базу},label=iIntegrator]
using System.Collections.Generic;
using ATF.Scripts.Helper;
namespace ATF.Scripts.Integration.Interfaces
{
	public interface IAtfIntegrator : IAtfGetSetRecordName
	{
		void SetUris(IEnumerable<string> filePaths);
		void Integrate();
		void IntegrateAndReplace();
		void IntegrateAll();
		void SaveUris();
		IEnumerable<string> LoadUris();
	}
}
\end{lstlisting}
\newpage
\section{Интерфейс модуля упаковки хранилища данных}
\begin{lstlisting}[caption={Интерфейс модуля упаковки хранилища данных},label=iPacker]
using System.Collections.Generic;
using ATF.Scripts.Storage.Utils;
using UnityEngine;
namespace ATF.Scripts.Storage.Interfaces
{
	public interface IAtfPacker
	{
		List<Record> Pack(Dictionary<string, Dictionary<FakeInput, Dictionary<object, AtfActionRleQueue>>> input);
		Dictionary<string, Dictionary<FakeInput, Dictionary<object, AtfActionRleQueue>>> Unpack(Slot slot);
		string ValidatePacked(List<Record> packed);
	}
}
\end{lstlisting}
\newpage
\section{Пример изменения тела метода и параметра класса Input в классе-арбитре ATFInput}
\begin{lstlisting}[caption={Пример изменения тела метода и параметра класса Input в классе-арбитре ATFInput},label=atf_methods_refactor_example]
//CSharp 7.0
public static bool simulateMouseWithTouches
{
	get => Intercept(Input.simulateMouseWithTouches, 	FakeInput.SIMULATE_MOUSE_WITH_TOUCHES, "Simulate mouse with touches");
	set => Input.simulateMouseWithTouches = value;
}
//CSharp 6.0
public static bool simulateMouseWithTouches
{
	get { return Intercept(Input.simulateMouseWithTouches, 	FakeInput.SIMULATE_MOUSE_WITH_TOUCHES, "Simulate mouse with touches"); }
	set { Input.simulateMouseWithTouches = value; }
}
...
//CSharp 7.0
public static bool GetMouseButton(int button) => Intercept(Input.GetMouseButton(button), FakeInput.GET_MOUSE_BUTTON, button);

//CSharp 6.0
public static bool GetMouseButton(int button)
{
	return Intercept(Input.GetMouseButton(button), FakeInput.GET_MOUSE_BUTTON, button); ...
\end{lstlisting}
\chapter{Алгоритмы}
\section{Работа инициализатора решения}
\begin{algorithm}
	\caption{Работа инициализатора решения}\label{alg:initializer}
	\begin{algorithmic}
		\State $allSystems \gets IAtfInitializable[3]$
		\State $allSystems[0] \gets Recorder.Instance$
		\State $allSystems[1] \gets Storage.Instance$
		\State $allSystems[2] \gets Saver.Instance$
		\ForAll{system in allSystems} \State{system.Initialize()} \EndFor
		\State $DependencyInjector.InjectIntoNamespace('ATF')$
	\end{algorithmic}
\end{algorithm}
\newpage
\section{Перехват и симуляция для Input.anyKeyDown}
\begin{algorithm}
	\caption{Перехват и симуляция для Input.anyKeyDown}\label{alg:input}
	\begin{algorithmic}
		\Procedure{AtfInput.anyKeyDown}{}
		\State \textbf{return} Intercept(Input.anyKeyDown, 
		\State GetCurrentFakeInput())
		\EndProcedure
		\Procedure{Intercept}{RealInput, FakeInput}
		\State \textbf{dynamic} result := RealOrFakeInputOrRecord(RealInput, FakeInput)
		\If{result is string}
		\State \textbf{throw} new Exception(\$"Deserialization failed: \{result\}")
		\EndIf
		\State \textbf{return} result
		\EndProcedure
		\Procedure{RealOrFakeInputOrRecord}{RealInput, FakeInput}
		\If{is Recording}
		\State Recorder.Record(RealInput)
		\State \textbf{return} RealInput
		\ElsIf{is Playing}
		\State \textbf{return} FakeInput
		\Else
		\State \textbf{return} RealInput
		\EndIf
		\EndProcedure
	\end{algorithmic}
\end{algorithm}