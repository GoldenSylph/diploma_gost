\nocite{*}
%\bibliographystyle{gost780u}
%\bibliography{0-main}
\begin{thebibliography}{9} 
	
	\bibitem{selenium} Vila E., Novakova G., Todorova D. Automation Testing Framework for Web Applications with Selenium WebDriver: Opportunities and Threats // Proceedings of the International Conference on Advances in Image Processing (ICAIP 2017). 2017. P. 144-150.
	
	\bibitem{rpc_testing} Sinaga A. M., Adi Wibowo P., Silalahi A., Yolanda N. Performance of Automation Testing Tools for Android Applications // 10th International Conference on Information Technology and Electrical Engineering (ICITEE). 2018. P. 534-539.
	
	\bibitem{appium_opencv} Mozgovoy M., Pyshkin E. Unity Application Testing Automation with Appium and Image Recognition // In: Itsykson V., Scedrov A., Zakharov V. (eds) Tools and Methods of Program Analysis. TMPA 2017. Communications in Computer and Information Science. 2017. №779. P. 139--150.
	
	\bibitem{wimp} Andries van Dam. Post-WIMP user interfaces // Commun. ACM. 1997. №2. P. 63–67.
	
	\bibitem{habr} Лучший игровой движок по версии пользователей хабра // Хабр URL: https://habr.com/ru/post/307952/ (дата обращения: Апрель 11, 2020).
	
	\bibitem{disser} Кугуракова В.В. Математическое и программное обеспечение многопользовательских тренажеров с погружением в иммерсивные виртуальные среды: дис. ... канд. тех. наук: 05.13.11. Казань, 2019. 187 с.
	
	\bibitem{vr-simulators} Kugurakova V.V. Automated approach to creating multi-user simulators in virtual reality // CEUR Workshop Proceedings. 2018. №2260. P. 313-320.  
	
	\bibitem{visual-editor} Kugurakova V.V., Abramov V.D., Abramsky M.M., Monks N., Maslaviev A. Visual editor of scenarios for virtual laboratories // DeSE 2017, Developments in the design of electronic systems.. Paris: Conference Publication Services (CPS), 2017. P. 242-247.
	
	\bibitem{biolab} Abramov V.D., Kugurakova V.V., Rizvanov A.A., Abramskiy M.M., Manakhov N.R., Evstafiev M.E., Ivanov D.S. Virtual Biotechnological Lab Development // BioNanoScience. 2017. №7. P. 363-365.
	
	%%% Вступление остортировано
	
	\bibitem{unity_input_systems} Unity -- Manual: Input // Unity Manual URL: https://docs.unity3d.com/Manual/Input.html (дата обращения: Апрель 12, 2020).
	
	\bibitem{unity_interface} Unity -- Scripting API: IPointerDownHandler // Unity Scripting API URL: https://docs.unity3d.com/2019.1/Documentation/ScriptReference/ EventSystems.IPointerDownHandler.html (дата обращения: Апрель 12, 2020).
	
	\bibitem{unity_is} Quick start guide // Unity Input System Manual URL: https://docs.unity3d.com/Packages/com.unity.inputsystem@1.0/manual /QuickStartGuide.html (дата обращения: Апрель 12, 2020).
	
	\bibitem{oreilly} Брюс П., Брюс Э. Практическая статистика для специалистов Data Science. СПб.: БХВ-Петербург, 2018. 304 с.
	
	\bibitem{solid} Мартин Р.С., Ньюкирк Д.В., Косс Р.С. Быстрая (гибкая) разработка программ на Java и C++: принципы, шаблоны, практика. М.: Издательский дом ``Вильямс'', 2001. 752 с.

	\bibitem{dry} Haoyu W., Haili Z. Basic Design Principles in Software Engineering // Fourth International Conference on Computational and Information Sciences. Chongqing: IEEE Computer Society, 2012. P. 1251-1254.
	
	\bibitem{razor} Blumer A., Ehrenfeucht A., Hasler D., Warmuth M.K. Occam's Razor // Information Processing Letters 24. 1987. №6. P. 377-380.
	
	\bibitem{dots} High-performance multithreaded stack of unity information-oriented technologies // unity.com URL: https://unity.com/ru/dots (дата обращения: Июль 19, 2019).
	
	\bibitem{ecs} Unity Foundation -- Introduction to ECS // unity3d.com URL: https://unity3d.com/ru/learn/tutorials/topics/scripting/introduction-ecs (дата обращения: Июль 19, 2019).
	
	\bibitem{interceptor} Schmidt D., Stal M., Rohnert H., Buschmann F. Pattern-Oriented Software Architecture // Patterns for Concurrent and Networked Objects. 2001. №2. P. 109-140.
	
	\bibitem{broker} Architectural pattern ``Broker'' // cs.uno.edu URL: http://cs.uno.edu/~jaime/Courses/4350/broker.ppt (дата обращения: Июль 19, 2019).
	
	\bibitem{pac} Coutaz J. PAC: An object oriented model for implementing user interfaces // ACM SIGCHI Bulletin. 1987. №19. P. 37-41.
	
	\bibitem{scrapy} Scrapy -- A Fast and Powerful Scraping and Web Crawling Framework // scrapy.org URL: https://scrapy.org/ (дата обращения: Апрель 17, 2020).
	
	\bibitem{jsr} JSR 299: Contexts and Dependency Injection for the JavaTM EE platform // jcp.org URL: https://jcp.org/en/jsr/detail?id=299 (дата обращения: Июль 19, 2019).
	
	\bibitem{atf_docs} ATF documentation // unity3dautotestframework.readthedocs.io URL: https://unity3dautotestframework.readthedocs.io/en/latest/?badge=latest (дата обращения: Апрель 17, 2020).
	
	\bibitem{virtual_trainers} Kugurakova V. V., Abramov V. D., Abramskiy M. M., Manakhov N., Maslaviev A. Visual editor of scenarios for virtual laboratories // 10th International Conference on Developments in eSystems Engineering (DESE 2017). Paris: IEEE, 2017. P. 242-247.
	
	\bibitem{rle} Смит С. Форматы сжатия данных // Электронные компоненты. 2009. №8. С. 83-87.
	
	\bibitem{git} GoldenSylph / Unity3DAutoTestFramework // github.com URL: https://github.com/GoldenSylph/Unity3DAutoTestFramework (дата обращения: Май 22, 2020).
	
	\bibitem{vrtk} VRTK - Virtual Reality Toolkit // vrtoolkit URL: https://vrtoolkit.readme.io/ (дата обращения: Май 18, 2020).
	
	\bibitem{video} Automated Test Framework Demo with DML BioLab // YouTube URL: https://youtu.be/YOUdXHlUIW4 (дата обращения: Май 21, 2020).
	
	\bibitem{open_source} Levine S.S., Prietula M.J. Open Collaboration for Innovation: Principles and Performance // Organization Science. 2013. №5. P. 1287-1571.

	\bibitem{assetstore} Automated Test Framework // Asset Store URL: https://assetstore.unity.com/packages/tools/utilities/\\automated-test-framework-167509 (дата обращения: Май 29, 2020).
	
	\bibitem{secr} Система автоматизации функционального тестирования для приложений на игровом движке Unity - Доклад SECR 2019 СПБ // SECR 2019 URL: https://2019.secrus.org/program/submitted-presentations/\\unity-engine-application-functional-testing-automation-based-on-native-tools/ (дата обращения: Май 29, 2020).
	
	\bibitem{prog_engine_journal} Авторам издательства ''Новые технологии`` // novtex.ru URL: http://www.novtex.ru/autor.htm (дата обращения: Май 29, 2020).
	
\end{thebibliography}